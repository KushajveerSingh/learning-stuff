\documentclass[12pt,a4paper]{article}
\usepackage[utf8]{inputenc}

\usepackage{mathtools}
\usepackage{amsmath}
\usepackage{amssymb}
\usepackage{amsthm}
\usepackage{amssymb}
\usepackage{mathdots}
\usepackage[pdftex]{graphicx}
\usepackage{fancyhdr}
\usepackage[margin=1in]{geometry}
\usepackage{multicol}
\usepackage{bm}
\usepackage{listings}
\usepackage{xcolor}
\usepackage{pdfpages}
\usepackage{algpseudocode}
\usepackage{tikz}
\usepackage{enumitem}
\usepackage[T1]{fontenc}
\usepackage{inconsolata}
\usepackage{framed}
\usepackage{wasysym}
\usepackage[thinlines]{easytable}
\usepackage{hyperref}
\usepackage{minted}
\usemintedstyle{perldoc}
\hypersetup{
    colorlinks=true,
    linkcolor=blue,
    filecolor=magenta,      
    urlcolor=blue,
}
\definecolor{codegreen}{rgb}{0,0.6,0}
\definecolor{codegray}{rgb}{0.5,0.5,0.5}
\definecolor{codepurple}{rgb}{0.58,0,0.82}
\definecolor{backcolour}{rgb}{0.95,0.95,0.92}
\lstdefinestyle{mystyle}{
    backgroundcolor=\color{backcolour},   
    commentstyle=\color{codegreen},
    keywordstyle=\color{magenta},
    numberstyle=\tiny\color{codegray},
    stringstyle=\color{codepurple},
    basicstyle=\ttfamily,
    breakatwhitespace=false,         
    breaklines=true,                 
    captionpos=b,                    
    keepspaces=true,                 
    numbers=left,                    
    numbersep=5pt,                  
    showspaces=false,                
    showstringspaces=false,
    showtabs=false,                  
    tabsize=4
}
\lstset{style=mystyle}
\newcommand\numberthis{\addtocounter{equation}{1}\tag{\theequation}}
\newcommand{\rightqed}{
\begin{flushright}
$\blacksquare$
\end{flushright}
}
\newcommand{\solution}{\noindent\textbf{Solution:}\\\indent}
\usepackage{graphics}
\usepackage{subfig}
\graphicspath{ {./images/} }

\title{CSCI 6470 Algorithms assignment 3}
\author{Kushajveer Singh}
\date{}

\begin{document}
\maketitle

% Start problem 1
\subsection*{Problem 1}
\textit{
    Fill DP table for $x=ACT$, $y=CAT$ and traceback path
}

\solution
(1) In case of a tie $\uparrow$ is used to break the tie. \\

\begin{tabular}{|c|c|c|c|c|}
\hline
& \textbf{0} & \textbf{C} & \textbf{A} & \textbf{T} \\
\hline
\textbf{0} & 0 & 0 & 0 & 0 \\
\hline
\textbf{A} & 0 & 0, $\uparrow$ & 1, $\nwarrow$ & 1, $\leftarrow$ \\
\hline 
\textbf{C} & 0 & 1, $\nwarrow$ & 1, $\uparrow$ & 1, $\uparrow$ \\
\hline
\textbf{T} & 0 & 1, $\uparrow$ & 1, $\uparrow$ & 2, $\nwarrow$ \\
\hline
\end{tabular} \\

(2) The traceback path is as follow

\noindent PrintLCS(P,x,3,3) = PrintLCS(P,x,2,2) + Print($x_2$) \\
PrintLCS(P,x,2,2) = PrintLCS(P,x,1,2) \\
PrintLCS(P,x,1,2) = PrintLCS(P,x,0,1) + Print($x_1)$ \\
PrintLCS(P,x,0,1) = return ()

Therefore the longest subsequence returned by PrintLCS is \textbf{AT}, where indexes are explored in the following order: $(3,3) \rightarrow (2,2) \rightarrow (1,2) \rightarrow (0,1)$.


\subsection*{Problem 2}
\solution
(1) The algorithm chooses $\uparrow$ i.e. $L(i-1,j)$ to break a tie.

(2) Alternate trace paths are as follows

Alternate trace path 1,
\begin{equation*}
    (7,6) \rightarrow (7,5) \rightarrow (6,4) \rightarrow (5,3) \rightarrow (4,3) \rightarrow (3,3) \rightarrow (2,2) \rightarrow (2,1) \rightarrow (1,0)
\end{equation*}
and the corresponding LCS printed by PrintLCS is \textbf{BCAB}.

Alternate trace path 2,
\begin{equation*}
    (7,6) \rightarrow (7,5) \rightarrow (6,4) \rightarrow (5,3) \rightarrow (5,2) \rightarrow (4,1) \rightarrow (3,0)
\end{equation*}
and the corresponding output of PrintLCS is \textbf{BDAB}.

\newpage
\subsection*{Problem 3}
\solution
(1) The main idea is to use a different symbol ($\uparrow_\leftarrow$) for the case when there is a tie. The modified LCS algorithm is as follows
\begin{lstlisting}[mathescape=true]
LCS(x,y)
    m = length(x), n = length(y);
    for i = 0 to m
        T[i,0] = 0
    for j = 0 to n
        T[0,j] = 0
    for i = 1 to m
        for j = 1 to n
            if $x_i$ = $y_j$
                T[i,j] = T[i-1,j-1]+1; P[i,j]='$\nwarrow$'
            else if T[i,j-1] > T[i-1,j]
                T[i,j] = T[i,j-1]; P[i,j]='$\leftarrow$'
            else if T[i,j-1] < T[i-1,j]
                T[i,j] = T[i-1,j]; P[i,j]='$\uparrow$'
            else
                T[i,j] = T[i-1,j]; P[i,j]='$\uparrow_\leftarrow$'
\end{lstlisting}

(2) The main idea is maintain a \textbf{lists} which is a list containing all longest common subsequence till (i,j). Now rather than printing the element in case of $\nwarrow$, we add the element to the right of every sublist of \textit{lists}. In case of $\uparrow_\leftarrow$, we combine the lists of (i-1,j) and (i,j-1). The modified PrintLCS algorithm is as follows
\begin{lstlisting}[mathescape=true]
PrintLCS(P,x,i,j, lists)
    if (i>0) $\wedge$ (j>0)                             $\longrightarrow c_1$
        if P[i,j] = '$\nwarrow$'                         $\longrightarrow c_2$
            lists = PrintLCS(P,x,i-1,j-1, lists) $\longrightarrow t(i-1,j-1)$
            Append $x_i$ to every sublist of lists  $\rightarrow \leq c_9length(lists)$
        else if P[i,j] = '$\leftarrow$'                    $\longrightarrow c_3$
            lists = PrintLCS(P,x,i,j-1, lists)   $\longrightarrow t(i,j-1)$
        else if P[i,j] = '$\uparrow$'                     $\longrightarrow c_4$
            lists = PrintLCS(P,x,i-1,j, lists)   $\longrightarrow t(i-1,j)$
        else if P[i,j] = '$\uparrow_\leftarrow$'                    $\longrightarrow c_5$
            list_1 = PrintLCS(P,x,i,j-1, lists)  $\longrightarrow t(i,j-1)$
            list_2 = PrintLCS(P,x,i-1,j, lists)  $\longrightarrow t(i-1,j)$
            lists = Merge list_1 and list_2      $\longrightarrow c_6$
        return lists                             $\longrightarrow c_7$
    else return ()                               $\longrightarrow c_8$
\end{lstlisting}

(3)\begin{equation*}
    t(i,j) = \begin{cases}
    a_1 & i=0\ or\ j=0 \\
    \max\begin{cases}
        t(i-1,j-1) + a_2length(lists) \\
        t(i,j-1) \\
        t(i-1,j) \\
        t(i,j-1)+t(i-1,j) + a_3
    \end{cases} & i>0\ and\ j>0
    \end{cases}
\end{equation*}

$a_1,a_2,a_3$ are some constants. \textit{length(lists)} is bounded as $\leq \max(2^{i+1}, 2^{j+1})$ as it is a constant time to append an element, but the number of elements in the list can grow exponentially in the worst case.

\subsection*{Problem 4}
\solution
At every index $(i,j)$, we will store two the value of $V(i,j)$ and the previous index which was used by the DP algorithm, which will be later used for the traceback process. \\

\begin{tabular}{|c|c|c|c|c|c|c|c|}
\hline
& \textbf{0}& \textbf{1}& \textbf{2}& \textbf{3}& \textbf{4}& \textbf{5}& \textbf{6}\\
\hline
\textbf{0}& 0& 0& 0& 0& 0& 0& 0\\
\hline
\textbf{1}& 0& 0 (0,1)& 2 (0,0)& 2 (0,1)& 2 (0,2)& 2 (0,3)& 2 (0,4)\\
\hline
\textbf{2}& 0& 7 (1,0)& 7 (1,1)& 9 (1,2)& 9 (1,3)& 9 (1,4)& 9 (1,5)\\
\hline
\textbf{3}& 0& 7 (2,1)& 7 (2,2)& 9 (2,3)& 10 (2,1)& 10 (2,2)& 12 (2,3)\\
\hline
\textbf{4}& 0& 7 (3,1)& 7 (3,2)& 8 (3,3)& 10 (3,4)& 16 (3,1)& 16 (3,2)\\
\hline
\end{tabular} \\

By tracing back from (4,6) we get the following path
\begin{equation*}
    (4,6) \rightarrow (3,2) \rightarrow (2,2) \rightarrow (1,1) \rightarrow (0,1)
\end{equation*}

and the highest value we get is 16, when we pack items 2 and 4.

\subsection*{Problem 5}
\solution

\begin{equation*}
    N(k,W) = \begin{cases}
    0 & k=0\ or\ W=0 \\
    N(k-1,W) & W<w_k \\
    \max\begin{cases}
        N(k-1, W-w_k)+1 \\
        N(k-1, W)
    \end{cases} & W\geq w_k
    \end{cases}
\end{equation*}
\end{document}

