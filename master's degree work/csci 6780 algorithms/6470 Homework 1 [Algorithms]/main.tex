\documentclass[12pt,a4paper]{article}
\usepackage[utf8]{inputenc}

\usepackage{mathtools}
\usepackage{amsmath}
\usepackage{amssymb}
\usepackage{amsthm}
\usepackage{amssymb}
\usepackage{mathdots}
\usepackage[pdftex]{graphicx}
\usepackage{fancyhdr}
\usepackage[margin=1in]{geometry}
\usepackage{multicol}
\usepackage{bm}
\usepackage{listings}
\usepackage{xcolor}
\usepackage{pdfpages}
\usepackage{algpseudocode}
\usepackage{tikz}
\usepackage{enumitem}
\usepackage[T1]{fontenc}
\usepackage{inconsolata}
\usepackage{framed}
\usepackage{wasysym}
\usepackage[thinlines]{easytable}
\usepackage{hyperref}
\usepackage{minted}
\usemintedstyle{perldoc}
\hypersetup{
    colorlinks=true,
    linkcolor=blue,
    filecolor=magenta,      
    urlcolor=blue,
}
\definecolor{codegreen}{rgb}{0,0.6,0}
\definecolor{codegray}{rgb}{0.5,0.5,0.5}
\definecolor{codepurple}{rgb}{0.58,0,0.82}
\definecolor{backcolour}{rgb}{0.95,0.95,0.92}
\lstdefinestyle{mystyle}{
    backgroundcolor=\color{backcolour},   
    commentstyle=\color{codegreen},
    keywordstyle=\color{magenta},
    numberstyle=\tiny\color{codegray},
    stringstyle=\color{codepurple},
    basicstyle=\ttfamily,
    breakatwhitespace=false,         
    breaklines=true,                 
    captionpos=b,                    
    keepspaces=true,                 
    numbers=left,                    
    numbersep=5pt,                  
    showspaces=false,                
    showstringspaces=false,
    showtabs=false,                  
    tabsize=4
}
\lstset{style=mystyle}
\newcommand\numberthis{\addtocounter{equation}{1}\tag{\theequation}}
\newcommand{\rightqed}{
\begin{flushright}
$\blacksquare$
\end{flushright}
}
\newcommand{\solution}{\noindent\textbf{Solution:}\\}

\title{CSCI 6470 Algorithms assignment 1}
\author{Kushajveer Singh}
\date{}

\begin{document}
\maketitle

% Start problem 1
\subsection*{Problem 1}
\textit{
    Recursive algorithm to find if $n \in$ VSN.
}

\solution
Let $n \in$ VSN ($n > 1$). $n$ is defined as follows 
\begin{align*}
    n &= 3*predecessor(n) + 5 \\
    n - 5 &= 3*predecessor(n) \numberthis \label{eq:q_1}
\end{align*}

From \eqref{eq:q_1}, we observe that $n - 5$ must be a multiple of 3. We will use this observation in the algorithm.

\begin{lstlisting}[title=Recursive algorithm to determine if $n \in VSN$, mathescape=true]
CHECK_VSN (n);
    // Stopping conditions
    if n < 1, return "No"
    if n = 1, return "Yes"
    
    // Recursively call CHECK_VSN
    temp = n - 5
    if temp%3 $\neq$ 0, return "No" //From$\eqref{eq:q_1}$, % is modulus operator
    else CHECK_VSN (temp/3)
\end{lstlisting}

\newpage
\subsection*{Problem 2 (1)}
\textit{
    What is functionality of algorithm $DoSomething$ do? How does it accomplish that?
}

\solution
It sort's a list in non-decreasing order. The explanation is as follows.

In Lines 2 FLAG is initalized to TRUE which means the code inside while loop would execute.

In Line 3 we start a $while$ loop. By default, $flag$ is set to $FALSE$ in the loop, which means if the loop has to start the next iteration the line 12 ($flag = TRUE)$ must be executed, which in turn will only take place if line 7 ($if (A[i-1] > A[i]$) is executed. This gives us the stopping condition of the while loop. The list $A$ must be sorted in non-decreasing order for the $while$ loop to stop.

Now consider the following list $A = B\circ max(A) \circ C$. Assume for now $B$ and $C$ are sorted lists of length $b$ and $c$ respectively and $max(A)$ is the maximum element in $A$ i.e. max element of $B$ and $C$ ($b + c + 1 = n)$.

\textbf{Claim:} One iteration of $while$ loop would shift $max(A)$ to the last index of list.

\textbf{Proof:} Expanding the list $A$, we get
\begin{equation*}
    \begin{matrix}
        B_1 & \hdots & B_b & max(A) & C_1 & C_2 & \hdots & C_c
    \end{matrix}
\end{equation*}

The first iteration of $while$ loop in $DoSomething$ will be as follows:
\begin{itemize}
    \item Index $2$ to $b+1$ $\rightarrow$ $if$ condition will always evaluate to false, as B is already sorted in non-decreasing order
    \item Index $b+2$ (i.e. at $C_1$) $\rightarrow$ In this case the $if$ condition would be evaluated, as $max(A) > C_1$. The result would be $swap(max(A), C_1)$ and we would get $max(A)$ at index $b+2$.
    \item Index $b+3$ (i.e. at $C_2$) $\rightarrow$ The previous step would repeat as $max(A) > C_2$, and $max(A)$ would be placed at index b+3
    \item Till last index $\rightarrow$ The above process would repeat till the last index. And in the end $max(A)$ would be placed in the last index.
\end{itemize}

Because the $if$ condition evaluated to $TRUE$ during execution, $flag$ would have value $TRUE$, meaning there would be another iteration of while loop. In this case the max element in index $[1, n-1]$ would be shifted to index $n-1$. And this process will repeat till $A$ becomes sorted in non-decreasing order.

We previously assumed that $B$ and $C$ are sorted, but now we consider the case when $B$ and $C$ are not sorted. In this case we can break the execution of the algorithm into two blocks as follows
\begin{itemize}
    \item List $B \rightarrow$ List $B$ can be recursively split into $B_1 \circ max(B) \circ B_2$ and the result would hold true.
    \item List $max(A) \circ C \rightarrow$ In this case, the $max(A)$ would be shifted to the last position as discussed in the last proof.
\end{itemize}

% In summary, the working of $DoSomething$ algorithm can be shown recursively as follows,
% \begin{align*}
%     \underbrace{
%     \begin{matrix}
%         A_1 & \hdots & A_{i-1}
%     \end{matrix}}_{\text{recursively split as $b \circ max \circ c$}}
%     \underbrace{\underbrace{\begin{matrix}
%         A_i
%     \end{matrix}}_{max(A)}
%     \begin{matrix}
%         A_{i+1} & \hdots & A_n
%     \end{matrix}
%     }_\text{$A_i$ would be shifted to index $n$}
% \end{align*}

\newpage
\subsection*{Problem 2 (2)}
\textit{
    Derive $T(n)$
}

\solution
The running time of every step of $DoSomething$ is shown below.

\begin{lstlisting}[title=Running time used by every step of $DoSomething$, mathescape=true]
DoSomething (A, n);
    n = length(A);                   $\longrightarrow c_1$
    flag = TRUE;                     $\longrightarrow c_2$
    while (flag) do                  $\longrightarrow \leq n\ times,\ each\ c_3$
    {
        flag = FALSE;                $\longrightarrow \leq n\ times, each\ c_4$
        for i = 2 to n do            $\longrightarrow \leq n\ times,\ each\ c_5$
            if (A[i-1] > A[i]) then  $\longrightarrow \leq n^2\ times,\ each\ c_6$
            {
                x = A[i-1];          $\longrightarrow \leq n^2\ times,\ each\ c_7$
                A[i-1] = A[i];       $\longrightarrow \leq n^2\ times,\ each\ c_8$
                A[i] = x ;           $\longrightarrow \leq n^2\ times,\ each\ c_9$
                flag = TRUE;         $\longrightarrow \leq n^2\ times,\ each\ c_{10}$
            }
    }
    return (A);                      $\longrightarrow c_{11}$
\end{lstlisting}
Summing up the running times used by all steps,
\begin{align*}
    T(n) &\leq c_1 + c_2 + nc_3 + nc_4 + nc_5 + n^2c_6 + n^2c_7 + n^2c_8 + n^2c_9 + n^2c_{10} + c_{11} \\
         &= n^2(c_6 + c_7 + c_8 + c_9 + c_{10}) + n(c_3 + c_4 + c_5) + c_1 + c_2 \\
         &= an^2 + bn + c \\
         &\leq dn^2 \text{ (for some d>0)} \numberthis \label{eq:q_2} 
\end{align*}

From \eqref{eq:q_2}, we get $T(n) = O(n^2)$.

\newpage
\subsection*{Problem 3}
\textit{
    Find $c$ and $k$
}

\solution
For base case $n=1$, we get
\begin{align*}
    T(1) &= 3(1)(\log1) + 4(1) - 2 \\
         &= 2
\end{align*}
As $T(n) \leq cn\log n$, we get
\begin{align*}
    T(1) &\leq c(1)(\log1) \\
    2 & \leq 0
\end{align*}
We get a contradiction, meaning it is a bad base case. So using $n=2$ as base case, we get
\begin{align*}
    T(2) &= 3(2)(\log2) + 4(2) - 2 \\
         &= 12
\end{align*}
Using this in $T(n) \leq cn\log n$, we get
\begin{align*}
    T(2) &\leq c(2)(\log2) \\
    12 &\leq 2c \\
    \implies & c \geq 6 \numberthis \label{eq:q_3}
\end{align*}
From \eqref{eq:q_3}, we get the value of $c = 6$ and $k = 2$.

\newpage
\subsection*{Problem 4}
\textit{
    Find T(n) with mergesort split into 3 lists.
}

\solution
Pseudo code for the modified merge sort is shown below
\begin{lstlisting}[mathescape=true]
MergeSort (x, start, end);
    if start = end: return x, start, end       $\longleftarrow c_1$
    else
        first_index = (start+end)*(1/3)        $\longleftarrow c_2$
        second_index = (start+end)*(2/3)       $\longleftarrow c_3$
        MergeSort(x, start, first_index)       $\longleftarrow T(n/3)$
        MergeSort(first_index+1, second_index) $\longleftarrow T(n/3)$
        MergeSort(second_index+1, end)         $\longleftarrow T(n/3)$
        Merge(x, start, end)                   $\longleftarrow \leq n$
\end{lstlisting}

The $Merge$ step (line 9) is linear because we can come up with $if-else$ conditions that can go though all possible outcomes and thus we would only require $O(1)$ time to fill an index of final array. As there are $n$ elements to be filled in the final array the time complexity becomes linear.

The base case for the algorithm is when we have only one element and it takes constant time to sort a single element.

The recurrence for the time function is
\begin{equation}
    T(n) = 
    \begin{cases}
    a & n = 1 \text{ ($a = c_1$)}\\
    3T(n/3) + n + b & n \geq 3 \text{ ($b = c_2 + c_3$)}\\
    \end{cases}
\end{equation}

\newpage
\subsection*{Problem 5}
\textit{
    Binarysearch with split at one-third.
}

\solution
The pseudo code for modified $BinarySearch$ is as follows
\begin{lstlisting}[title=Modified binary search, mathescape=true]
BinarySearch (x, i, j, key);
    if i > j: return 0                       $\longleftarrow c_1$
    else
        k = (i+j)*(1/3)                      $\longleftarrow c_2$
        if x[k] = key return k;              $\longleftarrow c_3$
        else
            if x[k] > key                    $\longleftarrow c_4$
                BinarySearch(x, i, k, key)   $\longleftarrow T(n/3)$
            else
                BinarySearch(x, k+1, j, key) $\longleftarrow T(2n/3)$
\end{lstlisting}

The recurrence relation for BinarySearch thus is
\begin{equation}
    T(n) = 
    \begin{cases}
    a & n = 1 \text{ ($a = c_1$)}\\
    max(T(n/3), T(2n/3)) + b & n \geq 2 \text{ ($b = c_2 + c_3 + c_4$)}\\
    \end{cases}
\end{equation}

BinarySearch is a deterministic algorithm, which means there is no randomness involved and it is safe to say that $T(n) >= T(k) \forall n > k$. This means $T(2n/3) >= T(n/3)$. The recurrence relation thus becomes,
\begin{equation}
    T(n) = 
    \begin{cases}
    a & n = 1\\
    T(2n/3) + b & n \geq 2\\
    \end{cases}
\end{equation}

Time time complexity of original binary search is $O(\log_{2} n)$. Here the base 2 is used because the array is split into two equal parts. In the modified binary search algorithm, the array is split into 2/3 (worst case), thus the time complexity becomes $O(\log_{3/2} n)$. But we can remove constants from logarithm as we are using Big-O notation, so the time complexity essentially remains same i.e. $O(\log n$) of the modified binary search algorithm.

\newpage
\subsection*{Problem 6}
\textit{
    Prove T(n) = O(n)
}

\solution

Given,
\begin{align*}
    T(n) = 
    \begin{cases}
    b & n \leq 4\\
    T(n/2) + T(n/4) + n + a & n \geq 5\\
    \end{cases}
\end{align*}

To prove, there exists some $c > 0$ for all $n \geq k$ s.t.
\begin{equation*}
    T(n) \leq cn
\end{equation*}

\textbf{Base case}:

\begin{itemize}
    \item $n=1$
    \begin{align*}
        T(1) &= b \leq c(1) \\
        \implies c &\geq b \numberthis \label{eq:q_6_1}
    \end{align*}
    \item $n=2$
    \begin{align*}
        T(2) &= b \leq c(2) \\
        \implies c &\geq b/2 \numberthis \label{eq:q_6_2}
    \end{align*}
    \item $n=3$
    \begin{align*}
        T(3) &= b \leq c(3) \\
        \implies c &\geq b/3 \numberthis \label{eq:q_6_3}
    \end{align*}
    \item $n=4$
    \begin{align*}
        T(4) &= b \leq c(4) \\
        \implies c &\geq b/4 \numberthis \label{eq:q_6_4}
    \end{align*}
\end{itemize}

From \eqref{eq:q_6_1}, \eqref{eq:q_6_2}, \eqref{eq:q_6_3}, \eqref{eq:q_6_4} we come to the conclusion
\begin{equation}
    c \geq b \text{ and } k = 1\label{eq:q_6_6}
\end{equation}

\textbf{Assumption}:

Assume the following,
\begin{align}
    T(n/2) &= O(n/2) \\
    T(n/4) &= O(n/4)
\end{align}

\textbf{Induction}:

\begin{align*}
    T(n) &= T(n/2) + T(n/4) + n + a \\
         &\leq c(n/2) + c(n/4) + n + a \\
         &= cn/2 + cn/2 - cn/2 + cn/4 + n + a \\
         &= cn - \underbrace{(cn/4 - n - a)}_{\text{$c>5$ makes this value $>0$}} \numberthis \label{eq:q_6_5}\\
         &\leq cn \\
         \implies T(n) &= O(n)
\end{align*}

In \eqref{eq:q_6_5}, if we set $c=5$ then the equation becomes $n/4 - a$. As n approaches infinity this term is positive.

In \eqref{eq:q_6_5}, we assumed $c > 5$ but in \eqref{eq:q_6_6} we got $c > b$. So a stronger inequality would be
\begin{equation}
    c \geq max(b, 5)
\end{equation}

Therefore, $T(n) \leq cn\ \ c \geq max(b,5)\ and\ n \geq 1$.

\newpage
\subsection*{Problem 7}
\textit{
    Modify code of Partition.
}

\solution
\begin{lstlisting}[title=Modified binary search, mathescape=true]
Partition (x, p, r);
    // Elements in [p, i] are less than pivot
    // Elements in [i+1, i+count] are equal to pivot
    // Remaining elements are greater than pivot
    
    x = A[r] // pivot element
    i = p-1 
    k = r-1
    count = 1 // number of duplicate items
    
    for j from 1 to k
        if A[j] < x
            i = i + 1
            exchange A[j] and A[i]
        if A[j] = x
            exchange A[j] and A[k]
            j = j - 1 // Need to check for A[j] element again
            k = k - 1
            count = count + 1
    
    // Move the last count elements to the middle       
    last = r
    for j from i+1 to i+count
        exchange A[j] and A[last]
        last = last - 1
    
    return i,i+count
\end{lstlisting}
\end{document}
