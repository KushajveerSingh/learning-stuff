\documentclass[12pt,a4paper]{article}
\usepackage[utf8]{inputenc}

\usepackage{mathtools}
\usepackage{amsmath}
\usepackage{amssymb}
\usepackage{amsthm}
\usepackage{amssymb}
\usepackage{mathdots}
\usepackage[pdftex]{graphicx}
\usepackage{fancyhdr}
\usepackage[margin=1in]{geometry}
\usepackage{multicol}
\usepackage{bm}
\usepackage{listings}
\usepackage{xcolor}
\usepackage{pdfpages}
\usepackage{algpseudocode}
\usepackage{tikz}
\usepackage{ulem}
\usepackage{enumitem}
\usepackage[T1]{fontenc}
\usepackage{inconsolata}
\usepackage{framed}
\usepackage{wasysym}
\usepackage[thinlines]{easytable}
\usepackage{hyperref}
\usepackage{minted}
\usemintedstyle{perldoc}
\hypersetup{
    colorlinks=true,
    linkcolor=blue,
    filecolor=magenta,      
    urlcolor=blue,
}
\definecolor{codegreen}{rgb}{0,0.6,0}
\definecolor{codegray}{rgb}{0.5,0.5,0.5}
\definecolor{codepurple}{rgb}{0.58,0,0.82}
\definecolor{backcolour}{rgb}{0.95,0.95,0.92}
\lstdefinestyle{mystyle}{
    backgroundcolor=\color{backcolour},   
    commentstyle=\color{codegreen},
    keywordstyle=\color{magenta},
    numberstyle=\tiny\color{codegray},
    stringstyle=\color{codepurple},
    basicstyle=\ttfamily,
    breakatwhitespace=false,         
    breaklines=true,                 
    captionpos=b,                    
    keepspaces=true,                 
    numbers=left,                    
    numbersep=5pt,                  
    showspaces=false,                
    showstringspaces=false,
    showtabs=false,                  
    tabsize=4
}
\lstset{style=mystyle}
\newcommand\numberthis{\addtocounter{equation}{1}\tag{\theequation}}
\newcommand{\rightqed}{
\begin{flushright}
$\blacksquare$
\end{flushright}
}
\newcommand\redsout{\bgroup\markoverwith{\textcolor{red}{\rule[0.5ex]{2pt}{0.4pt}}}\ULon}
\newcommand{\solution}{\noindent\textbf{Solution:}\\\indent}
\usepackage{graphics}
\usepackage{subfig}
\graphicspath{ {./images/} }

\title{Path and Cycle Graphs}
\author{Kushajveer Singh}
\date{}

\begin{document}
\maketitle

\subsection*{Problem 1}
\solution
Defining a new notation to make the solution cleaner
\begin{equation*}
    \begin{Bmatrix}
    x_1 \\
    x_2 \\
    \vdots \\
    x_v
    \end{Bmatrix}M_{P_v}
\end{equation*}
Here $x_i$ represent the diagonal values of $M_{P_v}$ and $M_{P_v}$ represents the adjacency matrix of $P_v$. Let $X$ be a $v\times v$ matrix that has -1 at index $(1,1)$ and $(v,v)$ and all other entries are 0.

\begin{align*}
    L_{C_{2v}} &= \begin{bmatrix}
    2 & \hdots & 0 \\
    \vdots & \ddots & \vdots \\
    0 & \hdots & 2
    \end{bmatrix} - \begin{bmatrix}
    M_{P_v} & X \\
    X & M_{P_v}
    \end{bmatrix} \\
    &= \begin{bmatrix}
    \begin{Bmatrix}
    2 \\
    \vdots \\
    2
    \end{Bmatrix}M_{P_v} & X \\
    X & \begin{Bmatrix} 2 \\ \vdots \\ 2\end{Bmatrix}M_{P_v}
    \end{bmatrix} \\
    L_{C_{2v}}\begin{bmatrix}I_v \\ I_v\end{bmatrix} &= \begin{bmatrix}\begin{Bmatrix} 1 \\ 2 \\ \vdots \\ 2 \\ 1 \end{Bmatrix}M_{P_v} \\\begin{Bmatrix} 1 \\ 2 \\ \vdots \\ 2 \\ 1 \end{Bmatrix}M_{P_v}\end{bmatrix}
\end{align*}
\begin{align*}
\begin{bmatrix}I_v & I_v \end{bmatrix}L_{C_{2v}}\begin{bmatrix}I_v \\ I_v\end{bmatrix} &= \begin{bmatrix}
    \begin{Bmatrix}
    2 \\ 4 \\ \vdots \\ 4 \\ 2
    \end{Bmatrix}M_{P_v}
    \end{bmatrix} \\
    &= 2 \begin{bmatrix}\begin{Bmatrix}
    1 \\ 2 \\ \vdots \\ 2 \\ 1
    \end{Bmatrix}M_{P_v}\end{bmatrix} \\
    &= 2L_{P_v}
\end{align*}
It is trivial to show that the right-hand side in above equation is equal to laplacian of $P_v$. As the path graph has diagonal values = (1, 2, ..., 2, 1) in the degree matrix and when we subtract adjacency matrix from degree matrix we are left with the above formula.
\rightqed

\newpage
\subsection*{Problem 2}
\solution
Showing for $\vec{y}_k(a)$
\begin{align*}
    \vec{y}_k(a+v) &= \sin(\pi k\frac{a+v}{v}) \\
                   &= \sin(\pi k\frac{a}{v} + \pi k) \\
                   &= \sin(\pi k\frac{a}{v}) \\
                   &= \vec{y}_k(a)
\end{align*}

Showing for $\vec{x}_k(a)$
\begin{align*}
    \vec{x}_k(a+v) &= \cos(\pi k\frac{a+v}{v}) \\
                   &= \cos(\pi k\frac{a}{v} + \pi k) \\
                   &= \begin{cases}
                   -\cos(\pi k\frac{a}{v}) & k=odd \\
                   \cos(\pi k\frac{a}{v}) & k=even
                   \end{cases} \\
                   &= \cos(\pi k\frac{a}{v}) \\
                   &= \vec{x}_k(a)
\end{align*}
\rightqed
\end{document}
